%
% File:     example.tex
%
% Document based on the template for thesis and dissertations
% developed by Steven White and Malcolm Hutson.
%
% Extensively modified by Adam Lewis (awlewis@cacs.louisiana.edu) to
% meet the UL Graduate School requirements as of Spring 2011. 
% 
% Moderately modified by Nicholas Lipari (nicholas.lipari@louisiana.edu) to meet UL Graduate School requirements as of Spring 2020.
%
% Unless otherwise expressly stated, this work is licensed under the
% Creative Commons Attribution-Noncommercial 3.0 United States License. To
% view a copy of this license, visit
% http://creativecommons.org/licenses/by-nc/3.0/us/ or send a letter to
% Creative Commons, 171 Second Street, Suite 300, San Francisco,
% California, 94105, USA.
%
% THE SOFTWARE IS PROVIDED "AS IS", WITHOUT WARRANTY OF ANY KIND, EXPRESS
% OR IMPLIED, INCLUDING BUT NOT LIMITED TO THE WARRANTIES OF
% MERCHANTABILITY, FITNESS FOR A PARTICULAR PURPOSE AND NONINFRINGEMENT.
% IN NO EVENT SHALL THE AUTHORS OR COPYRIGHT HOLDERS BE LIABLE FOR ANY
% CLAIM, DAMAGES OR OTHER LIABILITY, WHETHER IN AN ACTION OF CONTRACT,
% TORT OR OTHERWISE, ARISING FROM, OUT OF OR IN CONNECTION WITH THE
% SOFTWARE OR THE USE OR OTHER DEALINGS IN THE SOFTWARE.

\documentclass[12pt,dvipsnames]{report}	% The documentclass must be ``report''.

% Dissertation package style file.
\usepackage{template/uldiss}  		

% 
% Here is a collection of optional packages that can make your life
% far more pleasant while writing your thesis, prospectus, or
% dissertation.   You should tailor these to match the specific needs
% for your document.
%
\usepackage{amsmath,amsthm,amsfonts,amscd} % Some packages to write mathematics.
\usepackage{eucal} % Euler fonts
\usepackage{verbatim} % Allows quoting source with commands.
% The listings package supports the pretty-printing of source code.
% The listings packages supports the most commonly used programming
% languages.
\usepackage{listings}
\lstloadlanguages{Matlab,C++,C,Pascal}
%%\lstset{
%%%         language=Matlab,
%%         basicstyle=\footnotesize\ttfamily, 
%%         %numbers=left,              
%%         numberstyle=\tiny,          
%%         %stepnumber=2,              
%%         numbersep=5pt,              
%%         tabsize=2,                  
%%         extendedchars=true,         
%%         breaklines=true,            
%%         keywordstyle=textbf,    
%%         stringstyle=\ttfamily, 
%%         showspaces=false,       
%%         showtabs=false,         
%%         xleftmargin=17pt,
%%         framexleftmargin=17pt,
%%         framexrightmargin=5pt,
%%         framexbottommargin=4pt,
%%         %backgroundcolor=\color{lightgray},
%%         showstringspaces=false  
%% }
\lstset{
  %frame=tb,
  %language=Matlab,
  %aboveskip=2mm,
  %belowskip=2mm,
  showstringspaces=false,
  columns=flexible,
  basicstyle={\footnotesize\ttfamily},
  numbers=left,
  numberstyle=\tiny\color{gray},
  keywordstyle=\color{blue},
  commentstyle=\color{dkgreen},
  stringstyle=\color{mauve},
  breaklines=true,
  breakatwhitespace=true,
  tabsize=3,
  xleftmargin=17pt,
  %frame=single,
  %title=\lstname
}

%%%%%%%%%%%%%%%%%%%%%%%%%%%%%%%%%%%%%%%%%%%%%%%%%%%%%%%%%%%%%%%%%%%%%%
%%
%% The following disabled since it was overridden by caption setup in 
%% uldiss.sty (see line 1110)
%%
%%%%%%%%%%%%%%%%%%%%%%%%%%%%%%%%%%%%%%%%%%%%%%%%%%%%%%%%%%%%%%%%%%%%%%
% The caption package is used for fancy formatting of figure, table, and
% other captions.  It is very useful when combined with the listings package.
%%\usepackage{caption}
%%\DeclareCaptionFont{white}{\color{white}}
%%\DeclareCaptionFormat{listing}{\colorbox[cmyk]{0.43, 0.35, 0.35,0.01}{\parbox{\textwidth}{\hspace{15pt}#1#2#3}}}
%%\captionsetup[lstlisting]{format=listing,labelfont=white,textfont=white, singlelinecheck=false, margin=0pt, font={bf,footnotesize}}
%%

%%\usepackage{caption}
%%\DeclareCaptionFont{white}{\color{white}}
%%\DeclareCaptionFormat{verbatim}{\colorbox[cmyk]{0.43, 0.35, 0.35,0.01}{\parbox{\textwidth}{\hspace{15pt}#1#2#3}}}
%%\captionsetup[verbatim]{format=listing,labelfont=white,textfont=white, singlelinecheck=false, margin=0pt, font={bf,footnotesize}}

\usepackage{pdfpages}
%
% The graphicx package is the standard package for importing of graphics
% into LaTeX documents.   Note that we configure the package to, by
% default, look for PNG, JPEG, and PDF files in a sub-directory of the
% current directory.
\usepackage{graphicx}
\DeclareGraphicsExtensions{.png,.jpg,.pdf}
\graphicspath{{media/}}
%
% Use the subfig package for dealing with multiple part figures.
\usepackage[caption=false,labelfont=sf,textfont=sf,captionskip=5pt]{subfig}
% 
% The comment package is useful when you use EMACS for editing LaTeX
% documents.  The table editor in EMACS orgtbl-mode interfaces with this
% package for easy editing of tables using the org-mode table editing
% functions. 
\usepackage{comment}

%\usepackage{multirow}
%\usepackage{tikz}
%	\usetikzlibrary{trees, backgrounds}    
\usepackage{array}
\newcolumntype{P}[1]{>{\centering\arraybackslash}p{#1}}

%\usepackage[dvipsnames]{xcolor}

%\usepackage{draftwatermark}	% Uncomment this line to have the
				% word, "DRAFT," as a background
				% "watermark" on all of the pages of
				% of your draft versions. When ready
				% to generate your final copy, re-comment
				% it out with a percent sign to remove
				% the word draft before you run
				% latex for the last time.
%
% Document Type
%
% Choose a document type by commenting out every other type.
%\prospectus
\dissertation
%\masterthesis
%\masterreport


%
% School Customizations 
%
% Select your school
% If your school is not listed, this template has not been specifically  
%   customized for you yet, but the Grad School requirements will be met.
%   Try one that might be similar.

\cacscmps

%   ACM Transactions bibliography

%\cacseecs
%   ACM Transactions bilbiography

%\schoolofmusic
%   Chicago style bibliography with footnotes?

%
% Basic Information
%

\author{FirstName MI. LastName}
% Your name how it should normally appear across the document.
% The graduate school requires that your name always appear identically
%   every time that it is used. To help, we recommend use \theauthor
%   wherever your name should be printed for consistency.

\properauthor{LastName, FirstName MiddleName}
% Your proper name for alphabetizing. (Used in the abstract.)
% Last, First Middle Suffix

\title{A~Long~Title~With~Tildes~In~Between~The~Words}
% The title of your thesis/dissertation. Use a tilde (~) for any
%   spaces that should not be broken at line breaks.

\dean{Mary Farmer-Kaiser}
 % The Dean of the Graduate School

%\degree{Master of Science}
% The full title of your degree. 
% The default value is guessed by the document type and school.
  
%\major{Computer Science}
  % Your major.
  % The default value is guessed by your school.

%\graduationmonth{Spring}      
% Graduation semester, either Spring, Summer, or Fall, in the form
% as `\graduationmonth{Fall}'. Do not abbreviate.
% The default value (either Spring, Summer, or Fall) is guessed
% according to the time of running LaTeX.

% \graduationyear{2010} Graduation year, in the form as
% `\graduationyear{2001}'.  Use a 4 digit (not a 2 digit)
% number.  The default value is guessed according
%to the time of running LaTeX.
\graduationmonth{Spring}
\graduationyear{2020}

\previousdegrees{Degree-One, From-School, YEAR; Degree-Two, From-School, YEAR; Doctor of Philosophy, University of Louisiana at Lafayette, \thegraduationmonth \ \thegraduationyear}
% List all of your degrees, including the degree you are seeking with
% this document!

  
%
%
% Enter names of the member(s) of your committee. 
% Put one name per line with the name in square brackets. 
% The name on the last line, however, must be in curly braces.
%
% NOTE: The first member should be your supervisor.
%
% NOTE: Maximum six members. Minimum one member (supervisor).
%
\committeemembers
	[Member 1 (Chair)]
	[Member 2]
    [Member 3]
    {Member 4}
	
\committeememberstitle
	[Associate Professor of Computer Science]
	[Associate Professor of Computer Science]
	[Assistant Professor of Computer Science]
	{Assistant Professor of Computer Science}

%\supervisortitle{Dr.}   
  % Your supervisor's title (Dr., Mrs., Mr., Sir, etc)
  %
  % The default value is "Dr."

%
% Change the	Hyphenation behavior.
%
%
\hyphenation{FORTRAN Hy-phen-a-tion Position-Verification Intensity-Scaling Grid-Ratio-Intensity Position-Renderer}
% Manually specify how certain words should be hyphenated, if needed.
% You may add words without hyphens to request that they not be hyphenated.

%\hyphenpenalty=100000
% If you want no hyphenation in your document at all, uncomment
%   this line to set the hyphen penalty to an unreasonably
%   high value. 

%
% Some optional commands to change the document's defaults.
%
%
%\singlespacing
%\oneandonehalfspacing

%\singlespacequote
%\oneandonehalfspacequote

%\topmargin 0.125in	% Adjust this value if the PostScript file output
			% of your dissertation has incorrect top and 
			% bottom margins. Print a copy of at least one
			% full page of your dissertation (not the first
			% page of a chapter) and measure the top and
			% bottom margins with a ruler. You must have
			% a top margin of 1.5" and a bottom margin of
			% at least 1.25". The page numbers must be at
			% least 1.00" from the bottom of the page.
			% If the margins are not correct, adjust this
			% value accordingly and re-compile and print again.
			%
			% The default value is 0.125" 

	
%
%The document starts here.
%

\makeindex              % Make the index


\begin{document}
    
    \titlepage              % Produces the title page.
    
    \copyrightpage          % Produces the copyright page.
    
    \approvalpage           % Produces the approval page
    
    %
    % Dedication, epigraph, and/or acknowledgments are optional, but must
    % occur here.
    %
    %
    \begin{dedication}
    \null
\vfill
\textit{For ...}
\vfill
    \end{dedication}
    
    \begin{epigraph}
    \null
\vfill
Quotation
\begin{center}
%	\begin{tabular*}{\textwidth}{@{\extracolsep{\fill}}lr}
%    & \emph{Passages from the Life of a Philosopher (1864)}\\
%	& by Charles Babbage
%    \end{tabular*}
---\emph{Credit}
\end{center}
%\end{flushright}
\vfill


    \end{epigraph}
    
    \begin{acknowledgments}		% Optional
        I wish to thank ...\par


    \end{acknowledgments}
    
    % Table of Contents will be automatically generated and placed here.
    \tableofcontents   
    % List of Tables will be placed here, if applicable
    \listoftables      
    % List of Figures will be placed here, if applicable
    \listoffigures     
    
    %
    % Actual text starts here.%
    %
    % Including external files for each chapter makes this document simpler,
    % makes each chapter simpler, and allows for generating test documents
    % with as few as zero chapters (by commenting out the include statements).
    % You can even change the chapter order by merely interchanging the order
    % of the include statements.
    %
    %\include{chapter-introduction}
    
    \chapter{Introduction}
    \section{Section 1}
Text
    
    %\chapter{Background is a Very Long Chapter Title That Will Wrap Around, It's so Long That It's Still Going And Going}
    %\section{Fully-Enclosed Virtual Reality Spaces}
Lorem ipsum dolor sit amet, consectetur adipiscing elit. Duis tristique
nibh nec enim egestas lobortis. Cum sociis natoque penatibus et magnis
dis parturient montes, nascetur ridiculus mus. Praesent vitae lorem a
ante congue imperdiet. Aenean gravida lacus ac sem pharetra facilisis
laoreet elit facilisis. Integer ullamcorper blandit lorem, at pharetra
felis vestibulum a. Cras vitae elit non neque lacinia euismod eget ac
orci. Ut nibh ligula, porttitor eget facilisis id, fringilla eu
sapien. Vivamus in congue dui. Nunc lacus nunc, ornare eu vulputate
quis, auctor id lacus. Ut luctus interdum ligula. Donec ac rhoncus
urna. Ut suscipit nulla molestie libero pretium sed lacinia elit
commodo. Nulla a neque eget odio consequat bibendum. Aliquam egestas
sollicitudin eros at sollicitudin. Proin et ipsum at dolor molestie
rutrum vitae et leo.

\begin{table}[htb]
  \caption{Lorem ipsum dolor sit amet, consectetur adipiscing elit. Duis tristique nibh nec enim egestas lobortis.}
  \centering
  \begin{tabular*}{0.8\linewidth}{@{\extracolsep{\fill}}c|rrr}
    \hline
    \hline
    & \multicolumn{3}{c}{Delta (mm RMS)} \\
	Angle (deg) & Original & Offset & Quadtree \\
    \hline 
    90 & 0.488 & 0.465 & 0.477 \\
    45 & 4.932 & 5.180 & 0.882 \\
    30 & 8.148 & 8.051 & 2.452 \\
    15 & 20.327 & 19.258 & 5.233 \\
    \hline
	\hline
  \end{tabular*}
  \label{tbl:rotational-results}
\end{table}

Phasellus ante risus, porta quis pharetra ut, venenatis at neque. Duis
porttitor convallis dui, ornare lobortis justo congue eu. Sed lacinia
consectetur velit id posuere. Vestibulum ante ipsum primis in faucibus
orci luctus et ultrices posuere cubilia Curae; Etiam viverra venenatis
placerat. Nunc non quam arcu, ac gravida nibh. Duis dignissim ligula
eget nisl imperdiet pretium. Quisque non quam tellus. Maecenas ante
nisl, sollicitudin ut tincidunt sed, euismod et sem. Proin hendrerit,
nibh in sodales ultricies, enim purus varius neque, at mattis diam
mauris eget justo. Maecenas tristique ligula ut nunc iaculis
tincidunt. Nullam ut justo placerat nulla vehicula rutrum. Pellentesque
habitant morbi tristique senectus et netus et malesuada fames ac turpis
egestas.

In consectetur consectetur pharetra. Vivamus luctus nisl quam, vitae
egestas tortor. Nam turpis dui, dignissim iaculis faucibus fringilla,
laoreet eu leo. In libero metus, ullamcorper quis ullamcorper et,
porttitor vitae dolor. Sed eget faucibus odio. In nec diam vel quam
malesuada blandit vel at velit. Nulla nec lorem enim, ut imperdiet
nibh. In mi lorem, cursus eget blandit ut, luctus et dui. Duis dui
felis, porttitor eu rhoncus in, rhoncus sit amet lacus. Cras vel erat
lacus, id malesuada magna. Sed non augue sed lectus ullamcorper
eleifend. Cras nec odio purus, quis convallis elit. Ut condimentum
aliquet dolor in sodales. Phasellus id nisi purus. Vestibulum ante ante,
lobortis non semper vel, adipiscing ac risus. Ut placerat nibh eget nisl
mollis ut auctor massa accumsan. Nullam tempor, dui non malesuada
luctus, nisl augue dapibus ante, vel pharetra nisl turpis a tellus. Duis
molestie scelerisque blandit. Pellentesque nec rhoncus lectus.

Curabitur sapien enim, elementum sit amet mollis sed, convallis ac
enim. Nulla nunc tortor, semper eu volutpat sit amet, mattis in
eros. Quisque turpis risus, tincidunt nec euismod eu, aliquet vitae
massa. Ut gravida est a sem volutpat a auctor lorem dapibus. Lorem ipsum
dolor sit amet, consectetur adipiscing elit. Phasellus nec erat id erat
convallis cursus. Cras vitae fermentum tortor. Vivamus sit amet laoreet
nisi. Duis eu ipsum nec nunc blandit dictum. Integer mollis suscipit
justo, eget viverra eros feugiat vitae. Aliquam erat volutpat. Donec non
lectus lacus. Donec pulvinar eleifend risus, fringilla consectetur felis
dictum sed. Nam rhoncus lacinia odio, id sollicitudin diam placerat
quis. Integer quis feugiat quam. Morbi id neque sapien. Morbi massa
arcu, semper sit amet molestie eget, condimentum at nisl. Vestibulum
eget nulla nisl, eget scelerisque velit. Aenean volutpat mollis lectus,
vitae dapibus elit scelerisque nec.

  Phasellus in lorem in lectus egestas tincidunt. Pellentesque
  consectetur magna eget tellus pellentesque eleifend. Curabitur eget
  nulla ac mi laoreet eleifend. Nullam blandit nunc eget quam posuere
  mollis. Sed dignissim erat feugiat sem sollicitudin pellentesque
  hendrerit eros tincidunt. Donec velit sem, volutpat at consectetur at,
  rutrum sit amet tellus. Mauris augue odio, gravida sit amet suscipit
  ac, scelerisque in erat. Donec sit amet tellus ante, quis pretium
  ante. Integer et lacus sed leo ultrices mollis. Praesent scelerisque
  gravida tempus
\section{Head and Wand Tracking}
\section{Manipulation Devices}
    
    %\chapter{Related Work is also a Very Long Chapter Title That Will Wrap Around}
    \chapter{Literature Review}
    \label{ch:related_work}
\section{Some Papers}
Text
\section{Some Other Papers}
    
    %\chapter{Research Overview}
    %\section{Research Methodology}
\section{Architecture Overview}
    \chapter{Published Work}
    \label{chp:published_work}
\section{Introduction}
Text
    
    %\chapter{Study of Parametrized Triangle Model for Intensity and Position Interpolation in Irregular 2D Tactor Arrays}
    \chapter{Chapter One}
    \label{chp:research_1}
%\section{Abstract}
%
\section{Introduction}
Lorem ipsum dolor sit amet, consectetur adipiscing elit. Duis tristique
 
    
    \chapter{Chapter Two}
    \section{Introduction}
Lorem ipsum dolor sit amet, consectetur adipiscing elit. Duis tristique

    
    \chapter{Chapter Three}
    \section{Abstract}
Lorem ipsum dolor sit amet, consectetur adipiscing elit. Duis tristique
nibh nec enim egestas lobortis. Cum sociis natoque penatibus et magnis
dis parturient montes, nascetur ridiculus mus. Praesent vitae lorem a
ante congue imperdiet. Aenean gravida lacus ac sem pharetra facilisis
laoreet elit facilisis. Integer ullamcorper blandit lorem, at pharetra
felis vestibulum a. Cras vitae elit non neque lacinia euismod eget ac
orci. Ut nibh ligula, porttitor eget facilisis id, fringilla eu
sapien. Vivamus in congue dui. Nunc lacus nunc, ornare eu vulputate
quis, auctor id lacus. Ut luctus interdum ligula. Donec ac rhoncus
urna. Ut suscipit nulla molestie libero pretium sed lacinia elit
commodo. Nulla a neque eget odio consequat bibendum. Aliquam egestas
sollicitudin eros at sollicitudin. Proin et ipsum at dolor molestie
rutrum vitae et leo.

Phasellus ante risus, porta quis pharetra ut, venenatis at neque. Duis
porttitor convallis dui, ornare lobortis justo congue eu. Sed lacinia
consectetur velit id posuere. Vestibulum ante ipsum primis in faucibus
orci luctus et ultrices posuere cubilia Curae; Etiam viverra venenatis
placerat. Nunc non quam arcu, ac gravida nibh. Duis dignissim ligula
eget nisl imperdiet pretium. Quisque non quam tellus. Maecenas ante
nisl, sollicitudin ut tincidunt sed, euismod et sem. Proin hendrerit,
nibh in sodales ultricies, enim purus varius neque, at mattis diam
mauris eget justo. Maecenas tristique ligula ut nunc iaculis
tincidunt. Nullam ut justo placerat nulla vehicula rutrum. Pellentesque
habitant morbi tristique senectus et netus et malesuada fames ac turpis
egestas.

\section{Introduction}
\section{Previous Work}
\section{Hardware}
\section{Software}
\section{Testing}
\section{Discussion}
    
    \chapter{Chapter Four}
    \section{Abstract}
\section{Introduction}
\section{Previous Work}
\section{Hardware}
\section{Software}
\section{Testing}
\section{Discussion}
    
    \chapter{Conclusion}
    \section{Introduction}
Lorem ipsum dolor sit amet, consectetur adipiscing elit. Duis tristique

    
    \appendices
    
    \renewcommand\thesection{\Alph{section}}
    
    \section{Section One} 
Lorem ipsum dolor sit amet, consectetur adipiscing elit. Duis tristique

    
    \section{Section One} Lorem ipsum dolor sit amet, consectetur adipiscing elit. Duis tristique

    %
    % Generate the bibliography.
    %
    
    \nocite{*}      % This command causes all items in the               %
                    % bibliographic database to be added to              %
                    % the bibliography, even if they are not             %
                    % explicitly cited in the text.                      %
    
    % Here the bibliography is inserted.
    % Replace "example" with the name of your ".bib" file
    \bibliography{example}                      
    %\index{sensation@\emph{Sensation}}
    %
    
    
    %
    % Generate the index.
    % 
    %
    %\printindex     % Include the index here. Comment out this line      
    %               % with a percent sign if you do not want an index .  
    %
    
    % begin and end tags for abstract contained within, since the 
    % word count is found at the top of the same file
    \abstractwordcount{333}
% The number of words in your abstract.
% Unfortunately there is no clean way to count words in a section in Latex.
 
%% Remember to update the call \abstractwordcount{...}
\begin{abstract}
   Lorem ipsum dolor sit amet, consectetur adipiscing elit. Duis
   tristique nibh nec enim egestas lobortis. Cum sociis natoque penatibus
   et magnis dis parturient montes, nascetur ridiculus mus. Praesent
   vitae lorem a ante congue imperdiet. Aenean gravida lacus ac sem
   pharetra facilisis laoreet elit facilisis. Integer ullamcorper blandit
   lorem, at pharetra felis vestibulum a. Cras vitae elit non neque

\end{abstract}
    
    \begin{biography}
The [author's name] was born in [place].\par \end{biography}

\end{document}
% The following comment block is used by the different flavors of EMACS and
% the AUCTEX package to manage multiple documents.  In order for AUCTEX
% to understand you're working with multiple files, you should define
% the TeX-master variable as a file local variable that identifies your
% master document.
%
% Please do not remove.
%%% Local Variables: 
%%% mode: latex
%%% TeX-master: "example.tex"
%%% End: 
